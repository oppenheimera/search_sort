\documentclass[12pt, letterpaper]{article}
\usepackage[utf8]{inputenc}
\usepackage[margin=0.8in]{geometry}
\usepackage{layout}
\usepackage{dirtytalk}
\begin{document}
\setcounter{section}{-1}
\section{Sort}
\subsection{Insertion Sort}
Insertion sort works by scanning from right to left and when an item is out of place, \say{bubbling} that item to the left until it is in its rightful place.
\begin{itemize}
\item stable? Yes.
\item time complexity:
Worst case is $\theta n^2$.
\item space complexity: 
$\theta 1$ as everything is done in place.
\item best-case input:
A nearly-sorted array, as almost all needed work is done. 
\item worst-case input: An array in reverse order. $\sum_{k=1}^{n} = 1 + 2 + 3 + ... + n$
\end{itemize}
\subsection{Selection Sort}
Selection sort scans through the entire \emph{unsorted} part of the list looking for the next smallest item. Therefore the running-time of this algorithm is given by the triangular series $\sum_{k=1}^n k$ where $k$ is the number of compares.

\subsection{HeapSort}
When done in the classic \say{in place} method, HeapSort breaks items into single item queues and then merges those already \textit{sorted} queues into larger sorted queues
\subsection{MergeSort}
\subsection{QuickSort}
\section{Search}
\subsection{Djikstra's}
\subsection{A*}
\subsection{Primm's}
\section{Addenda}
\subsection{Asymptotics}

\begin{equation}
Taking:
\frac{r(\alpha)}{r(\beta)} = \bigg(\frac{\alpha}{\beta}\bigg)^b as\ \gamma = \delta
\end{equation}
\begin{equation}
we\ can\ siplify\ to:\ log_\gamma \delta = b \textrm{ and solve for $b$}
\end{equation}


\end{document}